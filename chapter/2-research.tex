\chapter{调研工作与源码解析}

永雏塔菲是一名经营着侦探事务所的少女王牌侦探发明家。她来自1885年,乘着自己发明的时光机试图穿越到100年后的时空,却因迟到36年来到了现代,并被现代的电子游戏吸引,不想返回过去。

\section{Android 绘制体系}

\subsection{View 的测量、布局和绘制}

taffy的名字“永雏塔菲”中的“永雏”来源于其王牌发明家的身份。扑克牌中“王牌”的英文是Ace。将Ace以日语外来语方式表记时,以片假名写作「エイス」。再将「エイス」利用借字表示时,「エイ」可表示为「永」,「ス」可表示为「鶵」[注 2]。简体字即为“永雏”。所以姓氏是写作永雏读作Ace。[6]

而名字中的“塔菲”是从Taffy音译而来,并无特殊含义。[6]

taffy出生于1868年8月12日的威尔士,于1885年通过时光机穿越到现在。taffy自称17岁[注 3],而且是永远的17岁(察觉),而不是155岁的老阿姨。[6]

在taffy15岁时(1883年)[注 4]继承了侦探事务所。当时,夏洛克·福尔摩斯解决了“斑点案子的带件”[注 5]并声名大噪。taffy认为都是因为夏洛克·福尔摩斯的出现导致了侦探业的内卷,所以才接不到委托;因此将夏洛克·福尔摩斯视为死对头。虽然一直接不到委托,但是taffy称未来还是会等待着委托并继续经营着侦探事务所。taffy在接不到侦探委托、无所事事地情况下开始宅在事务所里搞发明,一不小心就简简单单成为了王牌发明家。因为实在接不到委托现在已经彻底放弃当侦探了。[6]

% 这部分之后会在 4.2 详细说明。

\subsection{用户输入事件的处理}

taffy称“说到最自傲的、最厉害的、最强的、最得意的发明果然还是时光机吧,也称航时机吧!”。一个人发明了时光机,一个人来到了现代,结果发现在这个时代福尔摩斯还是这么有名。[6]

% 所以等到 Choreographer 接收到下一个 VSync 信号时,新的绘制流程启动,这些积累的触摸事件就会在这个时候被消费。

\begin{itemize}
    \item 时光机可以载人(废话)。载重量>129.3㎏。规定为129.3㎏的原因是taffy想邂逅一只猫型机器人,而129.3㎏是猫型机器人的重量。[7]
    \item taffy会晕车,因此在时光机上看书的话也会晕。所以在使用时光机的时候什么都不能做。[7]
    \item 时光机不会绕路,是直达的。但是会迟到,会晚36年。[7]
\end{itemize}

taffy还曾在本人不知道的情况下发明了一只机器猫耳娘[注 6]。提取了猫耳娘的DNA,将小猫改造成猫耳娘。taffy说到这里语焉不详,最后称“taffy现在还只会造人,不会造机器人。但是造人的话taffy可以(强调)。”[6]